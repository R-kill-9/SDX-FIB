\documentclass[a4paper, 10pt]{article}

\usepackage[utf8]{inputenc}

\makeatletter
\renewcommand\subsection{\@startsection{subsection}{2}{\z@}
	{-3.25ex\@plus -1ex \@minus -.2ex}%
	{1.5ex \@plus .2ex}%
	{\normalfont\large\itshape}}
\makeatother

\title{Reading Report:  Neville-Neil16}
\author{\textbf{Ricard Medina Amado}}
\date{\normalsize\today{}}

\begin{document}

\maketitle

\begin{center}
  Upload your report in PDF format.
  
  Use this LaTeX template to format the report, keeping the proposed headers.
  
  The length of the report must not exceed \textbf{5 pages}.
\end{center}

\section{Content}

\subsection{Identify the genre\protect\footnote{Genres: book, article, essay, report, review, manual, white paper, data sheet, weblog, etc.} of the document, its purpose, and its target audience.}

article

\subsection{Summarize the document, indicating the key concepts\protect\footnote{The summary should help you to answer the questions about the reading in the exam.}.}

Primerament s'explica la importància i rellevància de ls sincronització temporal entre els rellotges de diferents hosts, ja que d'això pot depndre el correcte funcionament d'un procès.
Són els cristalls de quars els encarregats de dita sincronització. Les CPUs són capaces d'executar-se a la velocitat que ho fan(3 GHz) degut a que a la placa base hi ha una pedra
de quars a la que se li aplica un corrent elèctric i aquesta vibra a una freqüència coneguda. Aquesta freqüència es reparteix per la resta del sistema.
Si un cristall s'escalfa, per exemple per l'excès de treball, el rellotge d'aquell host s'accelerarà i al revès si se li aplica fred.
Pera corregir aquestes desviacions, sobretot produïdes a hosts de baixa qualitat el 'SO' té diferents mecansimes per els quals fa les rectificaciones pertinents. Un altre mètodo possible és connectar al servidor un cristall
oscil·lador extern distribuit sobre un bus serial. Si s'injecta un senyal de bona qualitat a 10 MHz vis una linea serial també pot donar bons resultats. Distribuir aquesta senyalés viable per a un nombre petit de conexions(aprox. 50), però no més.


El temps actual sol ser invocat per els desenvolupadors de software amb la funció gettimeofday(). Encara que aquesta funció valdrà per a gairebé tots els sistemes clock\_gettime i
clock\_getres() poden oferir una millor flexibilitat. Els sistemes amb les rutines clock\_* ofereixen diferents tipus de relllotges per als programes d'usuari. Cada rellotge té dues 
variants. La primera, la precisió,  que obté el temps més acurat però tarda més en retornar-se i fast, que es retorna ràpidament amb un resultat menys acurat. En cas de voler 
calcular un interval temporal rdtsc és la millor opció.


Una solució per a trobar una bona font temporal és la reed, per exemple, es pot usar el Network Time Protocol(NTP). A NTP hi han rellotges amb diferents 'stratum'. A l'stratum 0 hi han
els que es consideren de referència, usats per GPS. a l'stratum 1 els rellotges estan connectats a l'stratum 0 via cable. A l'stratum 2 hi són connectats per cable a l'stratum 1 i així successivament fins al 15.
No hi ha cap estandar, només que quan més petit és el nombre d'stratum millor ha de ser la qualitat del rellotge.
Molts 'so' incorporen una crida anomenada adjtime() que permet a un programa extern com NTPd(NTP daemon) canviar el valor del rellotge. 
El Precision Time Protocol(PTP) va ser diseñat per a treballs on la necessitat de un a exactitut temporal de milisegons és essencial, actualment s'usa en fincances, en particular  a high-frequency trading(TDT).
El món HFT requereix d'una gran quantitat de servidors per a poder executar un gran nombre de transaccions en un temps exacte respectant l'ordre dels events. Les escales de segons han anat passant
de milisegons a microsegons i fins rangs més petits.


Per la seva part PTP difereix de NTP en molts punts, com que fou dissenyat per a pocs hosts en una única red, assumint que no hi han switchos ni hubs. Encara que es pot enrrutar la pèrdua de 
qualitat fa que no sigui útil. PTP està dissenya de forma multicast en la que hi ha un grandmaster, que és el millor rellotge i la resta són els seus esclaus. El grandmaster envia un paquet SYNC per segon amb el temps actual.
Els esclaus envien 'DELAY\_REQUEST' a les que el grandmaster respon amb 'DELAY\_RESPONSE'. Això permet a l'esclau determinar el seu delay i fer els canvis pertinents.
PTP assumeix simetria a la red aixì que suposa que hi ha el mateix delay de grandmaster a esclau que al revès. En cas de no ser-ho s'ofereix un daemon que pot ser configurat per arreglar dit prblema.
Els protocols temporals en red són molt sensibles al jiter. Executar el timing protocol a la seva propia red redueix el jitter creart per la propia red. El major contribuidor de jitter en PTP sol
ser el software. La proliferització de PTP al món HTF a permès que molts proveidors de tarjetes afegeixin oscil·ladors i hardware amb timestamping als seus dispositius.
Una pràctica comú per a la sincronitzacióals data centers és fer el rellotge del grandmaster bassat en GPS. Això surt més econòmic que posar un oscil·lador a cada server.


Encara que els primers data centers en aplicar PTP han estat els de l'aspecte financer, hi han moltes altres aplicacions que necessitaran una sincronització d'alta qualitat.

\section{Assessment}

\subsection{Rate the readability of the document: easy, readable, difficult, unreadable.}

readable

\subsection{Give your opinion of the reading assignment, indicating whether it should be included in next year's course or not.}

És interesant, l'únic aspecte negatiu  és que es fa una mica complicat llegir en mode article tal i com estàal pdf.
	
\end{document}



